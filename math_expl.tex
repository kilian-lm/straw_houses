\documentclass{article}
\usepackage{amsmath}

\begin{document}

\section{Estimating the Carbon Footprint Impact of Straw Houses}

Let's consider the following variables:
\begin{itemize}
	\item $C$: Total carbon footprint from all sources.
	\item $H$: Percentage of total carbon footprint caused by the housing industry.
	\item $S$: Number of straw houses, assuming exponential growth from the selected starting year.
	\item $R$: Reduction in carbon footprint per straw house.
	\item \texttt{starting\_year}: The year from which the exponential growth in straw houses begins (default 1950).
	\item \texttt{degradation\_rate}: The rate at which the carbon footprint reduction due to straw houses degrades over time.
	\item \texttt{growth\_rate}: The exponential growth rate of straw houses.
\end{itemize}

The total carbon footprint contributed by the housing industry can be represented as:
\[
C_{\text{housing}} = C \times \frac{H}{100}
\]

The number of straw houses grows exponentially from the selected starting year:
\[
S = (\text{year} \geq \text{starting\_year}) \times \exp(\text{growth\_rate} \times (\text{year} - \text{starting\_year}))
\]

The degradation of the reduction effect of straw houses is applied as:
\[
S \times= (1 - \text{degradation\_rate})^{(\text{year} - \text{starting\_year})}
\]

The total reduction in carbon footprint due to straw houses can be represented as:
\[
C_{\text{straw}} = S \times R
\]

Therefore, the adjusted carbon footprint for the housing industry, considering the subtractive effect of straw houses, is:
\[
C_{\text{adjusted}} = C_{\text{housing}} - C_{\text{straw}} = C \times \frac{H}{100} - S \times R
\]

This equation helps in understanding the potential impact of building houses with straw, reflecting a temporary decrease in CO2 emissions, considering the exponential growth of straw houses and the degradation of the reduction effect over time.

\end{document}
